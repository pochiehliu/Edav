\documentclass[]{article}
\usepackage{lmodern}
\usepackage{amssymb,amsmath}
\usepackage{ifxetex,ifluatex}
\usepackage{fixltx2e} % provides \textsubscript
\ifnum 0\ifxetex 1\fi\ifluatex 1\fi=0 % if pdftex
  \usepackage[T1]{fontenc}
  \usepackage[utf8]{inputenc}
\else % if luatex or xelatex
  \ifxetex
    \usepackage{mathspec}
  \else
    \usepackage{fontspec}
  \fi
  \defaultfontfeatures{Ligatures=TeX,Scale=MatchLowercase}
\fi
% use upquote if available, for straight quotes in verbatim environments
\IfFileExists{upquote.sty}{\usepackage{upquote}}{}
% use microtype if available
\IfFileExists{microtype.sty}{%
\usepackage{microtype}
\UseMicrotypeSet[protrusion]{basicmath} % disable protrusion for tt fonts
}{}
\usepackage[margin=1in]{geometry}
\usepackage{hyperref}
\hypersetup{unicode=true,
            pdftitle={GR5702 EDAV Homework 2},
            pdfauthor={Po-Chieh Liu (pl2441)},
            pdfborder={0 0 0},
            breaklinks=true}
\urlstyle{same}  % don't use monospace font for urls
\usepackage{color}
\usepackage{fancyvrb}
\newcommand{\VerbBar}{|}
\newcommand{\VERB}{\Verb[commandchars=\\\{\}]}
\DefineVerbatimEnvironment{Highlighting}{Verbatim}{commandchars=\\\{\}}
% Add ',fontsize=\small' for more characters per line
\usepackage{framed}
\definecolor{shadecolor}{RGB}{248,248,248}
\newenvironment{Shaded}{\begin{snugshade}}{\end{snugshade}}
\newcommand{\KeywordTok}[1]{\textcolor[rgb]{0.13,0.29,0.53}{\textbf{#1}}}
\newcommand{\DataTypeTok}[1]{\textcolor[rgb]{0.13,0.29,0.53}{#1}}
\newcommand{\DecValTok}[1]{\textcolor[rgb]{0.00,0.00,0.81}{#1}}
\newcommand{\BaseNTok}[1]{\textcolor[rgb]{0.00,0.00,0.81}{#1}}
\newcommand{\FloatTok}[1]{\textcolor[rgb]{0.00,0.00,0.81}{#1}}
\newcommand{\ConstantTok}[1]{\textcolor[rgb]{0.00,0.00,0.00}{#1}}
\newcommand{\CharTok}[1]{\textcolor[rgb]{0.31,0.60,0.02}{#1}}
\newcommand{\SpecialCharTok}[1]{\textcolor[rgb]{0.00,0.00,0.00}{#1}}
\newcommand{\StringTok}[1]{\textcolor[rgb]{0.31,0.60,0.02}{#1}}
\newcommand{\VerbatimStringTok}[1]{\textcolor[rgb]{0.31,0.60,0.02}{#1}}
\newcommand{\SpecialStringTok}[1]{\textcolor[rgb]{0.31,0.60,0.02}{#1}}
\newcommand{\ImportTok}[1]{#1}
\newcommand{\CommentTok}[1]{\textcolor[rgb]{0.56,0.35,0.01}{\textit{#1}}}
\newcommand{\DocumentationTok}[1]{\textcolor[rgb]{0.56,0.35,0.01}{\textbf{\textit{#1}}}}
\newcommand{\AnnotationTok}[1]{\textcolor[rgb]{0.56,0.35,0.01}{\textbf{\textit{#1}}}}
\newcommand{\CommentVarTok}[1]{\textcolor[rgb]{0.56,0.35,0.01}{\textbf{\textit{#1}}}}
\newcommand{\OtherTok}[1]{\textcolor[rgb]{0.56,0.35,0.01}{#1}}
\newcommand{\FunctionTok}[1]{\textcolor[rgb]{0.00,0.00,0.00}{#1}}
\newcommand{\VariableTok}[1]{\textcolor[rgb]{0.00,0.00,0.00}{#1}}
\newcommand{\ControlFlowTok}[1]{\textcolor[rgb]{0.13,0.29,0.53}{\textbf{#1}}}
\newcommand{\OperatorTok}[1]{\textcolor[rgb]{0.81,0.36,0.00}{\textbf{#1}}}
\newcommand{\BuiltInTok}[1]{#1}
\newcommand{\ExtensionTok}[1]{#1}
\newcommand{\PreprocessorTok}[1]{\textcolor[rgb]{0.56,0.35,0.01}{\textit{#1}}}
\newcommand{\AttributeTok}[1]{\textcolor[rgb]{0.77,0.63,0.00}{#1}}
\newcommand{\RegionMarkerTok}[1]{#1}
\newcommand{\InformationTok}[1]{\textcolor[rgb]{0.56,0.35,0.01}{\textbf{\textit{#1}}}}
\newcommand{\WarningTok}[1]{\textcolor[rgb]{0.56,0.35,0.01}{\textbf{\textit{#1}}}}
\newcommand{\AlertTok}[1]{\textcolor[rgb]{0.94,0.16,0.16}{#1}}
\newcommand{\ErrorTok}[1]{\textcolor[rgb]{0.64,0.00,0.00}{\textbf{#1}}}
\newcommand{\NormalTok}[1]{#1}
\usepackage{graphicx,grffile}
\makeatletter
\def\maxwidth{\ifdim\Gin@nat@width>\linewidth\linewidth\else\Gin@nat@width\fi}
\def\maxheight{\ifdim\Gin@nat@height>\textheight\textheight\else\Gin@nat@height\fi}
\makeatother
% Scale images if necessary, so that they will not overflow the page
% margins by default, and it is still possible to overwrite the defaults
% using explicit options in \includegraphics[width, height, ...]{}
\setkeys{Gin}{width=\maxwidth,height=\maxheight,keepaspectratio}
\IfFileExists{parskip.sty}{%
\usepackage{parskip}
}{% else
\setlength{\parindent}{0pt}
\setlength{\parskip}{6pt plus 2pt minus 1pt}
}
\setlength{\emergencystretch}{3em}  % prevent overfull lines
\providecommand{\tightlist}{%
  \setlength{\itemsep}{0pt}\setlength{\parskip}{0pt}}
\setcounter{secnumdepth}{0}
% Redefines (sub)paragraphs to behave more like sections
\ifx\paragraph\undefined\else
\let\oldparagraph\paragraph
\renewcommand{\paragraph}[1]{\oldparagraph{#1}\mbox{}}
\fi
\ifx\subparagraph\undefined\else
\let\oldsubparagraph\subparagraph
\renewcommand{\subparagraph}[1]{\oldsubparagraph{#1}\mbox{}}
\fi

%%% Use protect on footnotes to avoid problems with footnotes in titles
\let\rmarkdownfootnote\footnote%
\def\footnote{\protect\rmarkdownfootnote}

%%% Change title format to be more compact
\usepackage{titling}

% Create subtitle command for use in maketitle
\newcommand{\subtitle}[1]{
  \posttitle{
    \begin{center}\large#1\end{center}
    }
}

\setlength{\droptitle}{-2em}

  \title{GR5702 EDAV Homework 2}
    \pretitle{\vspace{\droptitle}\centering\huge}
  \posttitle{\par}
    \author{Po-Chieh Liu (pl2441)}
    \preauthor{\centering\large\emph}
  \postauthor{\par}
    \date{}
    \predate{}\postdate{}
  

\begin{document}
\maketitle

\subsubsection{\texorpdfstring{1.
\textbf{Flowers}}{1. Flowers}}\label{flowers}

Data: \texttt{flowers} dataset in \textbf{cluster} package

\begin{enumerate}
\def\labelenumi{(\alph{enumi})}
\tightlist
\item
  Rename the column names and recode the levels of categorical variables
  to descriptive names. For example, ``V1'' should be renamed
  ``winters'' and the levels to ``no'' or ``yes''. Display the full
  dataset.
\end{enumerate}

\begin{Shaded}
\begin{Highlighting}[]
\KeywordTok{data}\NormalTok{(flower)}
\KeywordTok{names}\NormalTok{(flower) =}\StringTok{ }\KeywordTok{c}\NormalTok{(}\StringTok{"winters"}\NormalTok{,}\StringTok{"shadow"}\NormalTok{,}\StringTok{"tubers"}\NormalTok{,}\StringTok{"color"}\NormalTok{,}\StringTok{"soil"}\NormalTok{,}\StringTok{"preference"}\NormalTok{,}\StringTok{"height"}\NormalTok{,}\StringTok{"distance"}\NormalTok{)}
\NormalTok{flower[}\DecValTok{1}\OperatorTok{:}\DecValTok{3}\NormalTok{]<-}\KeywordTok{as.data.frame}\NormalTok{(}\KeywordTok{ifelse}\NormalTok{(flower[}\DecValTok{1}\OperatorTok{:}\DecValTok{3}\NormalTok{] }\OperatorTok{==}\StringTok{ }\DecValTok{0}\NormalTok{, }\StringTok{"No"}\NormalTok{, }\StringTok{"Yes"}\NormalTok{))}
\NormalTok{flower}
\end{Highlighting}
\end{Shaded}

\begin{verbatim}
##    winters shadow tubers color soil preference height distance
## 1       No    Yes    Yes     4    3         15     25       15
## 2      Yes     No     No     2    1          3    150       50
## 3       No    Yes     No     3    3          1    150       50
## 4       No     No    Yes     4    2         16    125       50
## 5       No    Yes     No     5    2          2     20       15
## 6       No    Yes     No     4    3         12     50       40
## 7       No     No     No     4    3         13     40       20
## 8       No     No    Yes     2    2          7    100       15
## 9      Yes    Yes     No     3    1          4     25       15
## 10     Yes    Yes     No     5    2         14    100       60
## 11     Yes    Yes    Yes     5    3          8     45       10
## 12     Yes    Yes    Yes     1    2          9     90       25
## 13     Yes    Yes     No     1    2          6     20       10
## 14     Yes    Yes    Yes     4    2         11     80       30
## 15     Yes     No     No     3    2         10     40       20
## 16     Yes     No     No     4    2         18    200       60
## 17     Yes     No     No     2    2         17    150       60
## 18      No     No    Yes     2    1          5     25       10
\end{verbatim}

\begin{enumerate}
\def\labelenumi{(\alph{enumi})}
\setcounter{enumi}{1}
\tightlist
\item
  Create frequency bar charts for the \texttt{color} and \texttt{soil}
  variables, using best practices for the order of the bars.
\end{enumerate}

\subsubsection{2. Minneapolis}\label{minneapolis}

Data: \texttt{MplsDemo} dataset in \textbf{carData} package

\begin{enumerate}
\def\labelenumi{(\alph{enumi})}
\item
  Create a Cleveland dot plot showing estimated median household income
  by neighborhood.
\item
  Create a Cleveland dot plot to show percentage of foreign born,
  earning less than twice the poverty level, and with a college degree
  in different colors. Data should be sorted by college degree.
\item
  What patterns do you observe? What neighborhoods do not appear to
  follow these patterns?
\end{enumerate}

\subsubsection{3. Taxis}\label{taxis}

Data: NYC yellow cab rides in June 2018, available here:

\url{http://www.nyc.gov/html/tlc/html/about/trip_record_data.shtml}

It's a large file so work with a reasonably-sized random subset of the
data.

Draw four scatterplots of \texttt{tip\_amount} vs. \texttt{far\_amount}
with the following variations:

\begin{enumerate}
\def\labelenumi{(\alph{enumi})}
\item
  Points with alpha blending
\item
  Points with alpha blending + density estimate contour lines
\item
  Hexagonal heatmap of bin counts
\item
  Square heatmap of bin counts
\end{enumerate}

For all, adjust parameters to the levels that provide the best views of
the data.

\begin{enumerate}
\def\labelenumi{(\alph{enumi})}
\setcounter{enumi}{4}
\tightlist
\item
  Describe noteworthy features of the data, using the ``Movie ratings''
  example on page 82 (last page of Section 5.3) as a guide.
\end{enumerate}

\subsubsection{4. Olive Oil}\label{olive-oil}

Data: \texttt{olives} dataset in \textbf{extracat} package

\begin{enumerate}
\def\labelenumi{(\alph{enumi})}
\item
  Draw a scatterplot matrix of the eight continuous variables. Which
  pairs of variables are strongly positively associated and which are
  strongly negatively associated?
\item
  Color the points by region. What do you observe?
\end{enumerate}

\subsubsection{5. Wine}\label{wine}

Data: \texttt{wine} dataset in \textbf{pgmm} package

(Recode the \texttt{Type} variable to descriptive names.)

\begin{enumerate}
\def\labelenumi{(\alph{enumi})}
\item
  Use parallel coordinate plots to explore how the variables separate
  the wines by \texttt{Type}. Present the version that you find to be
  most informative. You do not need to include all of the variables.
\item
  Explain what you discovered.
\end{enumerate}


\end{document}
